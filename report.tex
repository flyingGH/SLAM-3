\documentclass[english]{article}

%I think these are useful for pdftex compatibility. I bastardised an old latex report.
\usepackage[T1]{fontenc}
\usepackage[latin9]{inputenc}

\usepackage[a4paper]{geometry}

%for manual adjustment of borders if necessary
%\geometry{verbose,tmargin=1.5cm,lmargin=2.5cm,rmargin=2.5cm}
\usepackage{fullpage}
\usepackage{verbatim}

\usepackage{amsthm}
\usepackage{amsmath}
\usepackage{amsfonts}
\usepackage{graphicx}
\usepackage{fixltx2e}
\usepackage{stfloats}
\usepackage{subfig}
\usepackage{babel}


%A quick note about style - for consistency, I propose using postscript for images and producing any mathematical graphs and figures in Mathematica. Also, make sure you add any figures, mathematica files or anything to git.
%I think you have more experience with LaTeX layout than me. I'm not so sure stylistically about things such as numbering of sections.

\begin{document}

%Working title? This will be incorporated into our final report, but right now it's just a mini-report. Since Kalman and Particle Filters are essentially just different implementations of Bayes' Filter, I think "Bayesian Filtering" is appropriate".
\title{Bayesian Filtering}

\author{Duncan Burke and Sebastian Pauka}
\maketitle

\section*{Introduction}

%Brief mention of Bayes' theorem. At what level are we meant to pitch this?

%Structure of the problem - x, u and z. Why this produces a HMM and the markov assumption.

\section*{Bayes' Filter}

%We need to cover this is some detail, I think. Fabio seems to be heavily interested in the theoretical side of robotics so we should at the very least justify and explain the algorithm and why it is impractical in its given form.

\section*{Kalman Filter}

%You wanted to do this, so I'll leave it to you. Remember, put anything about the Bayes' filter in general in the Bayes' filter section and leave a comparison of techniques to Discussion.
%Though I think it is still a good idea to go over the advantages and disadvantages in this section.

\section*{Particle Filter}


\section*{Discussion}


\end{document}
